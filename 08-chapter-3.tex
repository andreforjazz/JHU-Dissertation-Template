\chapter{3D imaging of ovarian cancer precursors in entire human Fallopian tubes} \label{chap:chap-3}
\begin{refsection}
     % Specify Chapter 2's `.bib` file here
    \addbibresource{references/chapter2/ref-extracts.bib}
    
    \section{Introduction}
    Ovarian cancer is the most lethal gynecological malignancy, with high grade serous carcinoma (HGSC) accounting for the majority of the cases \cite{Bowtell2015Rethinking,Kurman2016Dualistic}1–6. Accumulating evidence supports the fallopian tube, and serous tubal intraepithelial carcinoma (STIC), as the primary precursor of ovarian cancer6–15. This multi-organ progression from the fallopian tube to the ovary is unique among cancers, and its discovery has spurred research into the progression of STICs to invasive HGSC16–22. Yet, our current knowledge of ovarian cancer precursors largely stems from studies involving clinical specimens from individuals with ovarian cancer, gynecologic abnormalities, or from high-risk individuals possessing genetic risk factors23–26. Consequently, our knowledge of ovarian cancer precursors in wholly non-diseased specimens is limited27–30.
    STIC is diagnosed incidentally under microscope following the pathological criteria previously reported31,32. STIC lesions consist of atypical and multi-layered epithelial cells, with detectable mitotic figures and higher proliferative activity as compared to background epithelium. Alongside STIC, another related lesion emerges as a “p53 signature,” which is defined as a minute stretch of morphologically unremarkable epithelium but harboring TP53 mutations. The biological and clinical significance of p53 signatures is unclear and whether they represent the precursor lesions of STIC awaits further molecular studies. STICs are more commonly identified following the Sectioning and Extensively Examining the Fimbriated End, or SEE-FIM, protocol that has been adopted as a more thorough way in sampling fallopian tubes in clinical practice33–35. However, the diagnosis of STICs is solely based on 2D examination of tissue sections, and consequently, as little as 1\% of tubal tissues is microscopically examined by a pathologist as the bulk remains in archived tissue blocks36–39. As a result, the actual prevalence of STICs remains unknown and previous studies reported a wide range of STIC incidence, ranging from 11-61\% in HGSC patients40, 0-11.5\% in asymptomatic BRCA1/2 germline mutation carriers5,36,41,42, and <1\% in individuals without ovarian cancer or genetic risk factors43–46.  A significantly higher incidence was noted when the tissue blocks were flipped over and additional sections examined42, supporting the idea that current sampling lacks the sensitivity to exhaustively detect STIC lesions. Therefore, automated and exhaustive three-dimensional assessments are essential to resolve the spatial distribution and prevalence of rare and microscopic lesions, and to analyze their unique properties as the earliest stage of ovarian tumorigenesis 47–55.
    To address this gap, we developed a novel framework to comprehensively screen entire organ donor fallopian tubes for ovarian cancer precursors. Organ donation for scientific research is a precious resource that provides essential access to tissues unaffected by cancers and other abnormalities generally present in clinical specimens. Donor tissues have emerged as crucial for characterization of precancer frequency and molecular characteristics in organs including pancreas and colon56,57. Importantly, because detection of STICs requires consideration of H\&E, p53, and Ki67, existing 3D pathology workflows that rely solely on H\&E are insufficient to reliably identify these lesions. To overcome this, we developed a pipeline to combine H\&E-based cellular morphology with signal intensity from co-registered p53 and Ki67-stained IHC images to automatically highlight hundreds of potential precancerous lesions in a format easily reviewed by expert pathologists and amenable to further integration of multi-omics at regions of interest. This integration allowed an exhaustive and precise 3D mapping of microscopic p53 signatures, proliferative dormant and active STICs in whole human fallopian tubes at cellular resolution.
    While previous works have suggested these lesions are rare in low-risk populations43–45, using our automated and whole organ-scale workflow we find multiple p53 signatures, proliferative dormant or active STICs in all donor samples analyzed. Digital simulation of the SEE-FIM protocol in these donor organs explains their apparent elusiveness, revealing that the standard SEE-FIM protocol would detect less than half of the precursor lesions found here. To reduce the false-negative rate below 25\%, 150-250 equally spaced sections would be required. This is dramatically higher than the 10-20 sections typically analyzed in SEE-FIM and explains the historic lack of evidence of STIC lesions in non-diseased fallopian samples.
    Next, we further extended our workflow to integrate spatial proteomics, spatial transcriptomics, and spatial metabolomics to perform deeper molecular profiling specifically in regions within whole donor fallopian tubes that contained precancerous lesions. Using a 25-plex CODEX panel, we found that isolated STICs do not possess a unique immune microenvironment, unlike STICs found in the clinic, suggesting that immune evasion may not be an early hallmark in STIC progression. Using stimulated Raman scattering hyperspectral imaging (SRS-HSI) based spatial metabolomics approach58, we identified oxidative stress and increased rigidity that promotes malignant transformation. We also found increased nicotinamide adenine dinucleotide to oxidized flavin adenine dinucleotide (NADH/FAD) ratio in lesion cells compared to surrounding normal epithelial cells, suggesting the lesion epithelia subjects to oxidative stresses and rewires its metabolism towards glycolysis. Finally, integration of Visium spatial transcriptomics revealed significant and spatially confined upregulation of genes essential to cell proliferation, mitotic progression, and chromatin remodeling within the proliferative active STIC epithelium. Lastly, copy number alteration inference in proliferative active STIC showed chromosomal imbalances59. 
    In sum, through integration of high-resolution 3D imaging with molecular profiling, this study reveals the first detailed map of ovarian precancerous lesions in grossly unremarkable fallopian tubes and provides a framework for advancing the understanding of the earliest stages of ovarian cancer development.
    
    \section{Results}
    
    \subsection{Construction of cohorts of 3D-microanatomically labelled human fallopian tubes for assessment of ovarian cancer precursors }
    To assess the presence of ovarian cancer precancerous lesions, including p53 signatures and STICs, in gynecologically healthy women, we developed a pipeline to collect intact donor fallopian tube samples. Samples were procured through the network for pancreatic organ donors (nPOD) from individuals with no documented gynecological disease and no genetic risk factors for ovarian cancer (Fig. 1A). Organs were accepted if donors suffered no abdominal trauma and if warm ischemic time (WIT) was <16 h.
    Fallopian tubes were processed into formalin fixed, paraffin embedded (FFPE) blocks and exhaustively serially sectioned at a thickness of 4 microns. One in every two sections was stained with H\&E, one in every eight sections was IHC stained using p53, and one in every eight sections was IHC using Ki67 (Fig. 1B). Stained slides were imaged at 20x resolution (0.5 micron/pixel) using a Hamamatsu S210 scanner, stored as NDPI files, and post-processed into tiff image files. The mean number of sections cut for each human fallopian tube was 981, median 999, maximum 1373, and minimum 601. The average dimensions of the convoluted fallopian tubes in the FFPE blocks were 2.43 cm x 2.26 cm x 0.5 cm, median 2.34 cm x 2.10 cm x 0.5 cm. The average total volume per fallopian tube was 0.68 cm3, median 0.75 cm3. For context, the median volume of fallopian tube sampled by a single of whole slide is 0.00075 cm3 (=0.75cm3/999). The mean and median number of cells per fallopian tube was 438.3 million and 509 million, respectively. To preserve DNA, RNA, and proteins, unstained sections were mounted on plus slides and stored with desiccant packets at -20°C. 
    We trained three deep learning models to semantically segment the fallopian tube microanatomy. The first model segmented eight structures from the H&E-stained images: tubal epithelium, mesothelium, blood vessels, stroma, fat, nerve, rete ovarii, and background. The second model sub-classified the fallopian tubal epithelium into secretory and ciliated epithelial cells. The third model masked locations of positive p53 and Ki67 signals on the IHC images. Alignment of the H\&E and IHC segmented images into a volume via nonlinear image registration54,60 enabled automatic identification of secretory epithelial cells featuring p53+/Ki67+ and p53+/Ki67- signal. p53 staining positivity was defined herein as the staining pattern consistent with a TP53 missense mutation using the criteria previously reported61. Ki67 positivity was defined as the Ki67 labeling index was significantly higher than that of the adjacent or background epithelium. At these regions, we exported stacks of high-resolution registered 2D images, allowing human validation of detected lesions. A total of 1,285 deep learning-highlighted p53+/Ki67+ and p53+/Ki67- epithelium locations, with a mean of 257 and a median of 211 per fallopian tube, were automatically detected by our algorithm and then manually validated by pathologist experts. Highlighted locations were categorized as proliferative active STICs, proliferative dormant STICs, p53 signatures, or non-lesions (Fig. 1C). 
    Following detection of epithelial lesions, intervening unstained slides were used for deeper profiling. To understand the immune microenvironment of the proliferative active STIC, we applied a CODEX panel of 25 antibodies for WSI proteomics analysis (Fig. 1E). To understand the metabolic changes, we used spatial single-cell metabolomics (Fig. 1E). Lastly, to study gene expression variations and infer copy number alterations, we applied 10x Genomics Visium Cytassist (Fig. 1F).
    
    \subsection{3D characterization of the microanatomy of the human fallopian tube and STIC lesions}
    To comprehensively study the microanatomy of the fallopian tube in organ donor samples, we analyzed the results of the registered, segmented H\&E images (Fig. 2A). High-grade serous tumors primarily originate from secretory epithelial cells in the human fallopian tube62, highlighting the importance of understanding the composition and spatial arrangement of secretory epithelial cells in pre- and post-menopausal non-diseased fallopian tubes. Here, we analyzed 175.1 million pre-menopausal epithelial cells and 112.3 million post-menopausal epithelial cells. We produced z-projection heatmaps and 3D reconstructions, conveying the marked convolutions of the fallopian tube epithelial and the intermixing of secretory and ciliated epithelial cells (Fig. 2B and 2C). We found on average higher composition of secretory epithelial cells in post-menopausal (76\% secretory, 24\% ciliated) women compared to pre-menopausal women (58\% secretory, 42\% ciliated (Fig 2D). 
    Our 3D maps of whole fallopian tubes allowed us to computationally generate “virtual” sections of selected orientation (e.g. orthogonal to the main axis of the fallopian tube). To generate virtual sections along the length of the fallopian tube, we skeletonized each specimen by calculating the center path along the convoluted tubal lumen. At each cross section along the tube, we calculated the distance to the ovary (defined at the tip of the fimbriated end), and categorized this distance as proximal, medial, or distal. We visualized (Fig 2C, bottom; Supplementary Video 1) and quantified (Fig 2E, right) the distribution of secretory cells to show that the drop in ciliated cell content from pre- to post-menopausal primarily affects the locations on the fallopian tube medial and distal to the ovary, with similar composition of ciliated cells proximal to the ovary across age groups (Fig. 2E, right). We further quantified the secretory and ciliated cell composition as a function of distance along the center path, for precise sampling by generating thousands of orthogonal virtual cross sections along the epithelium (up to 10,255 virtual sections per sample). For each cross section, we quantified the overall frequency of secretory and ciliated epithelial cells from the distal isthmus to the proximal fimbriated end. 
    As the majority of STICs originate from secretory epithelial cells63, understanding their spatial distribution and age-associated changes is critical for understanding early ovarian tumorigenesis. Using our workflow to quantify the normal epithelial composition 3D, the data revealed that, in post-menopausal tubes, the proportion of secretory cells increases sharply toward the fimbriated end. In contrast, pre-menopausal tubes demonstrate a distal decrease in secretory cell percentage with a concomitant increase in total epithelial cell number due to expansion of ciliated cells. These data indicate that menopausal status substantially remodels the cellular composition of the distal tube towards a more secretory epithelial cell landscape, potentially influencing the local risk for neoplastic precursor lesions.
    
    \subsection{3D mapping of lesions in the non-diseased human fallopian tube epithelium}
    To implement a strategy for detailed 3D mapping of epithelial lesions in average-risk, nondiseased human fallopian tubes, samples were alternately stained with H\&E, p53 IHC and Ki67 IHC (Fig. 3A). Implementation of segmentation of the H\&E and IHC images allowed automated detection of p53 signatures, proliferative dormant STICs, and active STICs following standard clinical definitions (Fig. 3B). 3D volumetric renderings of these lesions convey their microscopic size and wide range  of 3D morphology (Fig. 3C; Supplementary Video 2). 
    We identified and 3D mapped 99 STICs, including 13 proliferatively active STICs, 86 proliferative dormant STICs, and 11 p53 signatures across 5 nondiseased whole human fallopian tubes from 5 distinct donors (Fig. 3D). According to menopausal status, we observed an average of zero STICs, 8.5 proliferative dormant STICs, 3 p53 signatures in pre-menopausal samples (Table S1, S2, S3, and S4). Our data revealed that ovarian precancerous lesions were present in 80\% of the examined fallopian tubes (Tabel S2, S6).  In post-menopausal samples, we observed an average of 4.33 STICs, 23 proliferative dormant STICs, 1.67 p53 signatures. Notably, one post-menopausal sample contained an unusual high number of lesions: 5 STICs, 37 proliferative dormant STICs, and 4 p53 signatures. The most common lesion we identified proliferative dormant STIC, and the most common location found to contain proliferative dormant STICs was the ampulla (55 lesions), compared to the fimbriated end (22 lesions) and isthmus (9 lesions). The most common location to contain proliferative active STICs was the fimbriated end (9 lesions) followed by the ampulla (4 lesions) and no active STICs were found in the Isthmus. We found p53 signatures in both the fimbriated end (5 lesions) and the ampulla (6 lesions).
    The occurrence of STIC, proliferative dormant STIC, and p53 signature lesions was higher in post-menopausal donors compared to pre-menopausal donors. STICs were detected in 67\% of post-menopausal donors but were absent in pre-menopausal cases. proliferative dormant STICs were detected in 100\% of post-menopausal donors and in 50\% of pre-menopausal donors (Table S5). The p53 signature was present in 67\% of post-menopausal and 50\% of pre-menopausal donors. When combining all donors, the overall prevalence was 40\% for STIC, 80\% for proliferative dormant STIC, and 60\% for p53 signature. 
    
    \subsection{Growth model of epithelial lesions in average-risk intact human fallopian tube samples}
    Our previous work on the mathematical modeling of precancerous lesions of the human pancreas (PanINs) demonstrated that a simple growth law, wherein each anatomically distinct lesion grows at a constant rate, fails to account for the very large lesions observed in our cohort. Explaining the size distribution required incorporating additional mechanisms, such as lesion splitting and merging. This hypothesis was supported by genomic data, which revealed that some large PanINs are composed of multiple clones that have collided within the pancreatic ductal system. In contrast, in the case of microscopic lesions in healthy fallopian tubes, the size distribution can be well explained by simple growth laws. A Kolmogorov-Smirnov test, maximizing the p-value, yielded \(V_\text{max} = 0.0605 \, \text{mm}^3\) and an exponent of \(\alpha = 1.63\) (\(p = 0.57\), Fig. 3F, right panel). This result suggests that, unlike PanINs, the lesions in this cohort of organ donor fallopian tube samples exhibit a lack of polyclonality (Fig. 3F, middle panel).
    
    \subsection{Development of virtual SEE-FIM for statistical determination of fallopian tube sampling guidelines}
    We asked why previous SEE-FIM-based assessments have not detected this high occurrence of lesions. To quantify the impact of subsampling when detecting ovarian cancer precursors, we virtually implemented a virtual SEE-FIM protocol. We generated longitudinal sections at the fimbriated end and transverse sections along the remainder of the ampulla and isthmus, as done in the clinic. To show the ability of SEE-FIM to identify lesions, including proliferative active STICs,  dormant STICS, and p53 signatures were highlighted in red, orange and yellow, respectively (Fig. 3G, and Fig. S2D; Supplementary Video 3).
    First, we simulated current SEE-FIM guidelines via collection of 20 equally spaced virtual sections, representing approximately 0.25\% volume of the entire organ. Within the extracted sections, SEE-FIM was able to identify 10.8\% of all lesions, 14.6\% of STICs, 10.1\% of proliferative dormant STICs, and 11.9\% of p53 signatures. These results reveal that conventional SEE-FIM protocol may significantly underestimate the true incidence of precursor lesions in human fallopian tubes. We determined that approximately 2.3\%, or 186 tissue sections, of the whole fallopian tube would need to be assessed to accurately identify all lesions with <25\% error (Fig. 3H, left panel, Fig. S2E, top left panel). Splitting by lesion type, we determined that to accurately identify STICs, proliferative dormant STICs, and p53 signatures with 25\% error, 1.8\% (149 sections), 2.4\% (190 sections), and 2.2\% (174 sections) of the fallopian tube would need to be assessed, respectively (Fig. 3H, Fig. S2E). 
    
    \subsection{Spatial protein marker profiling of STIC in non-diseased human fallopian tubes}
    To study the microenvironment surrounding the proliferative active STIC identified in 3D, we applied a panel of 25 protein markers using CODEX multiplexed imaging. We applied nucleus and cell body segmentation to identify 972,276 cells across the whole slide image (Fig.4A-B)66. We performed unsupervised clustering to obtain 30 distinct clusters, which we annotated and combined the clusters into 19 relevant cell phenotypes using previously established methods (Fig. 4C)67–69. These cellular phenotypes included STIC, epithelial cells, immune cell phenotypes (T cells, B cells, macrophages, neutrophils, dendritic cells), stromal cells (fibroblasts, smooth muscle cells), and tumor associated macrophages (TAMs), shown spatially in Fig. 4D. The protein expression matrix (Fig. 4E) and protein markers interactions70 (Fig. 4F) illustrate that epithelial and proliferating epithelial cells interact with EpCAM, Pan-CK, and Ki67. We labelled activated and memory T cells by CD3, CD4, CD8, and CD45RO, with additional links to IFNG and CD44. We identified B cells via CD20, and dendritic or APC populations by HLA-DR, CD11c, and CD141. Macrophages (TAMs) and monocytes associate with CD68 and IDO1, neutrophils with MPO, and endothelial cells with CD31. Mesenchymal and myofibroblast identities are confirmed by Vimentin and SMA. These specific interactions validate the correct phenotypic annotation in the dataset.
    Partition-based Graph Abstraction (PAGA) of single-cell proteomics in Fig. 4G71 showed interactions between the distinct cell phenotypes. STIC cells were closely associated with proliferating epithelial cells, supporting a trajectory consistent with malignant epithelial progression, while showing no direct connectivity to any other cell populations. Analysis of the PAGA connectivity map further identified interconnected immune cell populations comprising tumor-associated macrophages (TAMs), regulatory dendritic cells (DCs), activated T cells, and CD8+ memory cytotoxic T cells, suggesting potential immune coordination mechanisms. 
    PAGA analysis identified an immune network connecting TAMs, regulatory DCs, and activated T cells to CD8+ memory cytotoxic T cells. Additionally, CD8+ T cell connection to epithelial cells further bridging to STIC populations. This CD8+ T cell population's dual connectivity to both immunosuppressive cells and epithelial cells may indicate potential compromised immune surveillance72–74. While these topological relationships require functional validation, their organization suggests structurally relevant cellular interactions potentially governing STIC maintenance75. This immunological landscape closely mirrors established observations in ovarian cancer literature, where TAM enrichment and regulatory immune cell infiltration consistently correlate with tumor progression and poor clinical outcomes76–79. 
    To further validate the results obtained from the PAGA graphs, spatially resolved protein profiling was implemented to assess immunosuppressive expression within these interacting cell populations. Thus, to spatially assess the STIC microenvironment, STIC mask was generated and consequently dilated in 10-micron increments up to 500 microns from the STIC boundaries (Fig. S3B). For each distance interval computed, cellular composition was estimated and visualized (Fig. 4H). To quantify the differences in cellular composition relative to proximity to STIC location, comparison between regions close to the STIC (less than 100 microns) and regions distant to the STIC (between 100 and 500 microns) was performed. Proximity to the STIC showed modest enrichment in macrophages or monocytes, B cells, antigen-presenting cells (APCs), suppressed dendritic cells, and CD8 memory T cells (Fig. 4I).
    Spatial CODEX analysis identified an immunoregulatory microenvironment surrounding STIC lesions, with an increase of macrophages, regulatory dendritic cells, and CD8 memory T populations in proximal regions (Fig 4I). This validated the PAGA analysis, which showed direct connectivity between these immune populations and epithelial cell states. The spatial organization of these macrophage and CD8 T-cell populations aligns with the immune modulation observed in early high grade serous ovarian cancer76. These findings define the STIC microenvironment as a site of coordinated immune epithelial interactions that may facilitate early lesion persistence. Our results were consistent with those previously published80.
    
    \subsection{Spatial metabolomics profiling of STIC in non-diseased human fallopian tubes}
    Existing single cell and spatial transcriptomics data analysis has shown association of fallopian tube epithelium to genes and pathways associated with metabolic regulations in various contexts 81–83. With this in mind, we carefully evaluated unsaturated lipid level and redox ratio of preneoplastic epithelial lesions by multimodal two-photon stimulated Raman scattering imaging (SRS) (Figures 1A-C, S1A and S2A-C). We found that the unsaturated lipid level was reduced while the redox ratio indicated by NADH/FAD was increased in the lesion cells compared to the healthy surrounding cells (Figure 1C), suggesting the lesion epithelia subjects to ROS stress and metabolic remodeling towards glycolysis 84. Worth noting that the scattered distribution of NADH and FAD signals in lesions (Figure 1C in cyan and magenta) may indicate the fragmented mitochondria with compromised metabolic function.  We further applied hyperspectral SRS imaging and lipid subtype detection58 to perform an in situ lipidomic analysis and identified that lesions represented a distinct lipid profile manifesting in upregulated ceramide/PE and PC/PE ratio (Figures 1D-F and S1B), aligning with a study showing the disturbed homeostasis of ceramides and phospholipids in abnormal epithelial context 85. 
    Interestingly, Raman spectra showed distinct changes of lipid profile in different types of lesions (Figures S2D-F), suggesting the lipid metabolism is highly sensitive to the lesions and different lipid profile may represent the trajectory of lesion development. Altogether, our results provide new insights into the molecular mechanisms underlying the lesions of fallopian tube epithelium.
    
    Spatial transcriptomic profiling of STIC in nondiseased human fallopian tubes
    Using the CODA IHC-based deep learning method, we profiled STIC with spatial transcriptomics (Fig. 6A). STIC location was processed using Visium Cytassist for whole transcriptome profiling. Curation of the spatial spots identified STIC epithelial spots in red and non-STIC epithelial spots in green (Fig. 6B).
    Differential gene expressions of the proliferative active STIC against normal adjacent epithelium were obtained and shown in volcano plot (Fig. 6C). The upregulated genes in STIC, including KIF1A, TUBB2B, DLGAP5, BUB1, KIF2C, CDCA8, CDC20, CCNF, CCNB1, and PBK, suggest dysregulated cell cycle progression, mitotic spindle function, and chromosomal instability, which are seen in high-grade serous ovarian cancer86–96.  Immune-related genes like ULBP3 and BTNL2 may contribute to immune evasion, while JUN and NOX4 could promote survival and oxidative stress responses97–100. The presence of HNF4A, TFAP2A, and ADAM12 further supports a link to ovarian carcinogenesis through transcriptional deregulation, cellular differentiation, and extracellular matrix remodeling101–103. These findings reinforce STIC’s role as a precursor to high-grade serous carcinoma, with key drivers of malignancy already active104. Comparative analysis revealed significant upregulation of genes such as GPX2 (implicated in oxidative stress response) and HIST1H1D (a chromatin regulator) in STICs (Fig. 6D), mirroring patterns observed in advanced ovarian tumors105,106.
    Pathway analysis using the Hallmarks gene sets107, and the suppressed and activated pathways were computed (Fig. 6E-F). Hallmark pathway analysis revealed enrichment of several cancer-associated pathways in STICs, including spermatogenesis, G2M checkpoint, KRAS signaling, E2F targets, oxidative phosphorylation, and TNF$\alpha$ signaling via NFκB108–115. These activated pathways are consistent with clinical observations of early oncogenic signaling in STIC lesions that precede invasive high-grade serous carcinoma development116. Gene set enrichment analysis profiles confirmed significant enrichment of proliferation-associated pathways and G2M checkpoint genes (Fig. 6G), showing dysregulated cell cycle characteristics of both STICs and invasive ovarian cancers.
    To investigate chromosomal instability in proliferative active STIC, copy number analysis (CNA) was inferred from the spatial transcriptomics data (Fig. 6H-I, Fig. S6F), which revealed gains in chr6p22, chr6p21, chr1p32, and chr16p13, and losses in chr17p13, chr9q33, chr9q34, chr22q11, chr22q12, and chr22q13. These results align with clinical genomic studies showing that copy number alterations and genomic instability are early events in STIC lesions14,108,117,118. Chromosomal 6 gains and chromosomal 22 depletions were spatially located on the STIC (Fig. 6K). Notably, gains in chr6p, which harbors immune-related genes, have been linked to immune evasion and tumor progression in ovarian cancer, while losses in chr17p13, encompassing TP53 and are associated with impaired DNA damage response and genomic instability118. These alterations may collectively contribute to early malignant transformation and aggressive phenotypes in STIC lesions. 
    To further explore chromosomal alterations, we also applied inferCNV (Fig. S6F) and identified chromosomal gains in chromosomes 1, 6, 8, 16, and 19. Chromosomal losses were detected in chromosomes 4, 9, 13, 15, 17, 18, and 22.  Comparison to a large cohort study of 47 patients with proliferative active STICs108, which showed chromosomal gains in chromosomes 1, 2, 3, 6, 7, 8, 10, 12, 16, 19, and 20; and chromosomal depletions in chromosomes 4, 5, 6, 7, 8, 9, 11, 13, 15, 16, 17, and 22. Similarly, genes altered in these regions include TP53 (chr17p13), MYC (chr8q24.21), CCNE1 (chr19q12), CDKN2A/CDKN2B (chr9p21), BRCA1 (chr17q21), and NF2/TIMP3 (chr22q12-13).These chromosomal targets highlight pathways associated with cell cycle regulation, DNA repair, and immune modulation. Conversely, the large patient cohort study also reported unique gains in chr2, chr3, chr7, chr10, chr12, and chr20, and unique losses in chr5, chr7, chr8, and chr11, not observed in our nondiseased, average-risk donor cohort analysis. These regions encompass genes such as PIK3CA/MECOM (chr3q26), ETV6/FOXM1 (chr12p13), and APC (chr5q22), which are involved in PI3K signaling, transcriptional regulation, and tumor suppressor pathways. 
    
    \printbibliography[heading=subbibliography, title={References}]
\end{refsection}